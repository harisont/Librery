% !TeX spellcheck = IT
\documentclass[a4paper]{article}
\usepackage[utf8]{inputenc}
\usepackage[italian]{babel}
\usepackage{graphicx}
\usepackage{float} %per posizione assoluta delle immagini [H]
\usepackage[colorlinks=true,linkcolor=blue]{hyperref}
\usepackage{nameref} 
\graphicspath{{./images/}}
\def\code#1{\texttt{#1}}
\def\image[#1][#2]#3{
  \begin{figure}[H]
  \centering
  \includegraphics[#2]{#1}
  \caption{#3}
  \end{figure}}

\title{Librery: un'app per la gestione e la condivisione delle proprie letture}
\author{Arianna Masciolini}

\begin{document}
\maketitle
\newpage
\tableofcontents
\newpage
\section{Introduzione}
Librery è un'applicazione Android completamente libera ed open source volta alla gestione delle proprie letture. Essa permette di effettuare ricerche nel più grande indice di testi integrali esistente tramite le API RESTful di Google Libri e di salvare localmente i dati relativi ai libri letti o che s'intende leggere. L'utente può aggiungere delle annotazioni e la propria valutazione ai libri letti e, grazie ad un servizio web anch'esso realizzato nell'ambito del progetto Librery, mantenere un proprio "scaffale" pubblico online volto a condividere con altri bibliofili i propri libri preferiti.
	\subsection{Casi d'uso}
	Tutte le sopra citate funzionalità sono presenti in Google Libri, che è ormai non solo un motore di ricerca specializzato, ma un servizio web complesso e articolato collegato agli account Google dei propri utenti. \newline
	Librery, che effettua solamente richieste di dati pubblici a Google Libri e che dunque non necessita affatto un account Google, vuole proporsi come una più semplice ma valida alternativa per quanti preferiscono avere la possibilità di mantenere un certo grado di riservatezza sui propri dati personali, eventualmente indicando in modo esplicito i dati di quali volumi vadano pubblicati online.
\newpage
\section{Architettura}
L'architettura di Librery è piuttosto semplice, ma inusuale. 
Il dispositivo Android è client di due diversi server, uno di Google, al quale vengono effettuate le richieste JSON nel momento in cui si fa una ricerca, l'altro realizzato nell'ambito del progetto Librery, volto a gestire le collezioni pubbliche di libri dei diversi utenti, cui l'app invia parte dei dati salvati nel proprio database locale.
\image[diagram.png][scale=0.7]{Architettura del progetto Librery}
Nelle sezioni successive verranno descritti approfonditamente il client ed il server Librery online.
\newpage
\section{Il client: l'app Android}
In questa sezione si è scelto di concentrare l'attenzione sugli aspetti del client realizzato che hanno a che vedere con l'uso delle API di Google Libri e, più in generale, con i vari aspetti del networking, accennando solo in maniera molto breve alle soluzioni proposte per quel che riguarda la realizzazione dell'interfaccia utente e del database locale.
	\subsection{Visione d'insieme}
	Per la realizzazione dell'applicativo si è scelto il linguaggio Kotlin, totalmente interoperabile con Java e ormai ufficialmente supportato nel mondo Android, anche all'interno di Android Studio. La scelta è dovuta alla peculiare brevità e alla leggibilità del codice Kotlin ed ha permesso di sviluppare l'applicazione in tempi relativamente brevi.\newline
	Il database locale è realizzato non direttamente in SQLite ma, come raccomandato nella documentazione ufficiale di Android, mediante la libreria Room Persistence.
	%? Aggiornare activity a lavoro finito
	A livello di interfaccia utente, l'app è composta da quattro activity:
	\begin{itemize}
		\item \code{MainActivity}: schermata composta di due \code{Fragment} in cui è possibile visualizzare, sotto forma di \code{RecyclerView}, l'elenco dei libri letti e l'elenco dei libri da leggere, ossia i risultati di due diverse interrogazioni del database locale;
		\item \code{SearchActivity}: schermata tramite la quale l'utente può effettuare ricerche in Google Libri. Il risultato è una stringa in formato JSON, che per semplicità viene passata come tale a \code{SearchResultsActivity};
		\item \code{SearchResultsActivity}: qui avviene il parsing del JSON ottenuto in risposta dai server di Google Books ed i risultati vengono visualizzati in una \code{RecyclerView} del tutto analoga a quella impiegata in \code{MainActiviy};
		\item \code{ViewBookDetailsActivity}: permette di visualizzare i dettagli di un libro selezionato da una \code{RecyclerView}, modificarne alcuni attributi, ad esempio la valutazione, e salvarli localmente e/o online;
		\item \code{LoginActivity}: %?
		%? immagini schermate d'interesse (login, viewdetails, search, searchresults)
	\end{itemize}
	% room persistence database
	\subsection{Utilizzo delle API di Google Books}
	\subsection{Librerie}
		\subsubsection{OkHTTP}
		\subsubsection{Gson}
		\subsubsection{Picasso}
\newpage
\section{Il server per la condivisione online del proprio scaffale}

\end{document}