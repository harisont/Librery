%!TEX program = xelatex
\documentclass{beamer}
\logo{\includegraphics[height=0.8cm]{logo.png}\vspace{239pt}}
\usetheme{mhthm}
\graphicspath{{./images/}}
\title{Libre\textcolor{ProgBarBGColor}{ry}}
\subtitle{Un'app Android per gestire le proprie letture}
\author{Arianna Masciolini}
\institute{Università degli Studi di Perugia}
\date{Giugno 2018}
\setcounter{showSlideNumbers}{1}

\begin{document}
	\setcounter{showProgressBar}{0}
	\setcounter{showSlideNumbers}{0}
	\frame{\titlepage}
	\setcounter{framenumber}{0}
	\setcounter{showProgressBar}{1}
	\setcounter{showSlideNumbers}{1}
	\section{Intro}
	\begin{frame}
		\frametitle{Cos'è Librery}
		\begin{itemize}
			\item Applicazione Android per gestione di libri
			\item Funzionalità:
			\begin{itemize}
				\item Ricerca avanzata in Google Libri
				\item Salvataggio dati personali su database locale
				\item Possibilità di condivisione di consigli di lettura con gli altri utenti di Librery
			\end{itemize}
		\item Attenzione alla privacy: 
		\begin{itemize}
			\item Non è necessario un account Google
			\item Possibilità di diversificare dati privati ed eventuali dati pubblici
		\end{itemize}
		\end{itemize}
	\end{frame}
	\begin{frame}
		\frametitle{Architettura}
		\image[diagram.png][scale=0.5]
		Architettura del progetto Librery: un server e il client sono stati realizzati appositamente.
	\end{frame}
\section{Client}
	\begin{frame}
		\frametitle{Visione d'insieme}
		\begin{itemize}
			\item Kotlin
			\item Database locale: \textbf{Room Persistence Library}
			\item Utilizzo \textbf{API REST} di Google Libri
			\item Scambio di oggetti \textbf{JSON}
		\end{itemize}
	\end{frame}
	\begin{frame}
		\frametitle{Librerie}
		Librerie di terze parti:
		\begin{itemize}
			\item \textbf{OkHTTP}: client HTTP semplice ed efficiente
			\item \textbf{Picasso}: caricamento immagini
			\item \textbf{Gson}: parsing JSON
		\end{itemize}
	\end{frame}
	\begin{frame}
		\frametitle{Ricerca}
		\image[search.png][scale=0.3]
		Richiesta HTTP a Google Libri. Ricerca avanzata: \code{GET} con parametri specificati dall'utente tramite form.
	\end{frame}
	\begin{frame}
		\frametitle{Pubblicazione consigli}
		\image[share.png][scale=0.3]
		Chiamata \code{POST} al server Librery Online. I parametri sono recuperati dal database locale, ma modificabili prima di effettuare la richiesta HTTP.
	\end{frame}
	\begin{frame}
		\frametitle{Lista consigli}
		\image[recomm.png][scale=0.08]
		Dati recuperati in parte da Librery online, in parte da Google Libri tramite chiamate \code{GET}.
	\end{frame}
\section{Server}
	\begin{frame}
		\frametitle{Visione d'insieme}
		\begin{itemize}
			\item Hosting: Altervista
			\item Database MySQL
			\item Script in PHP per operazioni di \code{SELECT} ed \code{INSERT}
		\end{itemize}
	\end{frame}
	\begin{frame}
		\frametitle{Database \code{my\_anonimalettori}}
		\image[struct.png][scale=0.15]
		Struttura volutamente minimale del database MySQL: i dati relativi ai libri sono accessibili tramite un'altra chiamata REST tramite il loro id su Google Libri (campo \code{libro}).
	\end{frame}
	\begin{frame}
		\frametitle{Script PHP}
		\begin{itemize}
			\item \code{db\_connect.php} contiene i parametri di connessione al database
			\item \code{select\_all.php} per operazioni di \code{SELECT}
			\item \code{insert.php} per inserimento di singoli record (consigli) nel database
			\begin{itemize}
				\item uso di \textbf{prepared statement} per evitare SQL injection
			\end{itemize}
		\end{itemize}
	\end{frame}
\end{document}